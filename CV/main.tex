%-------------------------
% Resume in Latex
% Author : Jared Galloway
% License : MIT
%------------------------

\documentclass[letterpaper,11pt]{article}


\usepackage{latexsym}
\usepackage[empty]{fullpage}
\usepackage{titlesec}
\usepackage{marvosym}
\usepackage[usenames,dvipsnames]{color}
\usepackage{verbatim}
\usepackage{enumitem}
\usepackage[hidelinks]{hyperref}
\usepackage{fancyhdr}
\usepackage[english]{babel}
\usepackage{fontawesome5}

\usepackage{multibib}
\newcites{prim}{first/co-first author publications}
\newcites{supp}{supporting author publications}

\pagestyle{fancy}
\fancyhf{} % clear all header and footer fields
\fancyfoot{}
\renewcommand{\headrulewidth}{0pt}
\renewcommand{\footrulewidth}{0pt}

% Adjust margins
\addtolength{\oddsidemargin}{-0.5in}
\addtolength{\evensidemargin}{-0.5in}
\addtolength{\textwidth}{1in}
\addtolength{\topmargin}{-.5in}
\addtolength{\textheight}{1.0in}

\usepackage{xcolor}
\definecolor{mylightblue}{RGB}{70, 130, 180}
\definecolor{mygrey}{RGB}{124, 127, 134}
\usepackage{hyperref}
\hypersetup{
    colorlinks=true,
    linkcolor=mygrey,
    citecolor=black,
    urlcolor=mylightblue,
}
\urlstyle{same}
\let\oldhref\href
\renewcommand{\href}[2]{\oldhref{#1}{\bfseries#2}}

\raggedbottom
\raggedright
\setlength{\tabcolsep}{0in}

% Sections formatting
\titleformat{\section}{
  \vspace{-4pt}\scshape\raggedright\large
}{}{0em}{}[\color{black}\titlerule \vspace{-5pt}]

%-------------------------
% Custom commands

% make a minipage rule for personal statement
\newcommand{\personalstatement}[1]{
  \begin{center}
  \begin{minipage}[t]{0.96\textwidth}
    \linespread{0.90}\selectfont
    \vspace{.1cm}\small{
      #1 \vspace{1pt}
    }
  \end{minipage}
  \end{center}
}

\newcommand{\resumeNoWrapItem}[2]{
  \item\small{
    \textbf{#1}{: #2 \vspace{-2pt}}
  }
}

\newcommand{\resumeItem}[2]{
  \begin{minipage}[t]{0.9\textwidth}
    \linespread{0.85}\selectfont
    \vspace{.1cm}\item\small{
      \textbf{#1}{: #2 \vspace{1pt}}
    }
  \end{minipage}
}

\newcommand{\resumeSubheading}[4]{
  \vspace{-1pt}\item
    \begin{tabular*}{0.97\textwidth}[t]{l@{\extracolsep{\fill}}r}
      \textbf{#1} & #2 \\
      \textit{\small#3} & \textit{\small #4} \\
    \end{tabular*}\vspace{-3pt}
}

\newcommand{\resumeSoftwareSubheading}[3]{
  \vspace{-1pt}\item
    \begin{tabular*}{0.97\textwidth}[t]{l@{\extracolsep{\fill}}r}
      \textbf{#1}\label{#1} & #2 \\
    \end{tabular*}\vspace{-6pt}
    \begin{center}
    \begin{minipage}{0.87\textwidth}
    \linespread{0.80}\selectfont
        % \setlength{\parindent}{2em}
        % \begin{center}
        \small#3
        % \end{center}
    \end{minipage}\vspace{-5pt}
    \end{center}
}


\newcommand{\resumeSubItem}[2]{\resumeNoWrapItem{#1}{#2}\vspace{-4pt}}

\renewcommand{\labelitemi}{$\circ$}
\renewcommand{\labelitemii}{$\cdot$}

\newcommand{\resumeSubHeadingListStart}{\begin{itemize}[leftmargin=*]}
\newcommand{\resumeSubHeadingListEnd}{\end{itemize}}
\newcommand{\resumeItemListStart}{\begin{itemize}[]}
\newcommand{\resumeItemListEnd}{\end{itemize}\vspace{-5pt}}


%-------------------------------------------
%%%%%%  CV STARTS HERE  %%%%%%%%%%%%%%%%%%%%%%%%%%%%


\begin{document}

%----------HEADING-----------------
\begin{tabular*}{\textwidth}{l@{\extracolsep{\fill}}r}
  \textbf{\Large Jared Galloway} \\
  \textit{Curriculum vitae, circa} $`24$ \\
  &  \href{jaredgalloway07@gmail.com}{jaredgalloway07@gmail.com} \\
  &  \href{https://github.com/jgallowa07}{github.com/jgallowa07} \\
  &  \href{http://jaredgalloway.org/}{jaredgalloway.org} \\
  &  +1-406-579-6768
\end{tabular*}

%-----------PERSONAL STATEMENT--------

% TODO currently on LinkedIn?
% TODO currently on website?
% TODO currently on github?
% TODO currently on resume?
% TODO make actual bibliography of publications

% currently on CV
\section{Personal Statement}

\personalstatement{

% Being raised in Montana I have been fortunate enough to
% experience and foster appreciation for our natural world.
% This is what drew me to biological sciences and also plays out
% into much of my personal life as an avid skier, biker, and hiker.
% I find that my passion  has been a driving force
% in my career as a data scientist, as I am often drawn to projects
% that have a direct impact on the environment, or human health.

I am a Data Scientist with a
computer and information science (CIS) background,
and a focus on bioinformatics.
I have seven years of experience in an academic research environment 
conducting various flavors of 
methods development, implementation, and downstream analysis
in the biological sciences.
This has provided me with a strong intuition
for taking noisy empirical data through the various steps 
such as filtering out artifacts, normalizing,  modeling, and visualization
to produce tidy, coherent, and reproducible datasets.
I take pride in clearly communicating these results,
and very much enjoy the dicussion that follows.

% While my career has primarily rewarded insight and results 
% that inform our understanding of biological mechanisms, 
% I find that I'm inclined towards the design and engineering
% behind the research.
% I've found that my education in product design has 
% provided me with a unique perspective when visualizing data, 
% and designing software interfaces.
% In my life outside work I find real appreciation and and in inclination towards
% eloquent design of a product that feels natural to the end-user.

%I enjoy playing the game of detective with these datasets by
%visualizing, hypothesizing, and then leveraging the appropriate 
%statistical tests of significance.
% I am a strong communicator, and have a knack for teaching, 
% I have become skilled at writing reproducible data-processing workflows 
% which tie publicly available software 
% with "in-house" software written either by myself,
% or in collaborate efforts with students, post-Docs, and faculty.
% I'm fascinated by both the mechanisms that drive our world, and the way
% in which we can engineer solutions to problems that arise from these mechanisms.
% While a good chunk of the methods being performed in the pipeline are often heavily
% informed by either previous studies, or a math-inclined collegue, 
% I find that the process of implementing or retro-fitting these methods 
% has provided me with a strong intuition for the underlying mathematics and statistics.
% This can be seen in the methods I've  

}

%-----------EDUCATION-----------------
\section{Education}
    \resumeSubHeadingListStart
        \resumeSubheading
            {Master of Science, Biology}{University of Oregon -- Eugene, OR}
            {Knight Campus - Bioinformatics and Genomics Program}{June 2019 -- December 2020}
        \resumeSubheading
            {Bachelor of Science, Computer \& Information Science}{University of Oregon -- Eugene, OR}
            {College of Arts and Sciences - Departmental Honors - Minor in Product Design}{Aug 2013 -- July 2018}
    \resumeSubHeadingListEnd

%--------PROGRAMMING SKILLS------------
\section{Technical Skills}

\resumeSubHeadingListStart
    \resumeSubItem{Languages}
        {
            Python, R, C/C++, Bash, Java, Javascript
            html, \LaTeX, markdown, reStructured text
        }
    \resumeSubItem{Mathematics}
        {
            Advanced Statistics and Bayesian Modeling, 
            Linear Algebra, 
            Discrete Math, 
            Calculus
        }
    \resumeSubItem{Operating systems}
        {
            Linux, OSX, Windows
        }
    \resumeSubItem{Development}
        {
            Git, Travis CI, GitHub Actions, % version control 
            GNU make, cmake, scons, % build sysntems
            zshell/bash, vim8, tmux, % command line interface 
            Visual Studio Code, 
            SLURM-based HPC clusters, 
            AWS S3, % AWS
            Conda/Mamba, % dependency management 
            Docker, Apptainer (formerly Singularity) % containerization 
        }
    \resumeSubItem{Analysis}
        {
            JAX, Tensorflow/Keras, PyTorch, 
            Pandas, Polars, Numpy, Xarray, 
            Matplotlib, Plotnine (grammar of graphics), Altair (interactive visualization),
            scikit-learn, SciPy, BioPython,
        }
    \resumeSubItem{Workflow managers}
        {
            Nextflow, Snakemake
        }
\resumeSubHeadingListEnd

%-----------EXPERIENCE-----------------
    
    % Develop
    % Design
    % Drive
    % Evaluate
    % Implement
    % Manage
    % Build
    % Organize
    % Demonstrate
    % Reported
    % Visualize
    % Analyze X
    % Utilize
    % Leverage
    % Collaborate
    % Construct
    % Assist
    % Conduct
    % Lead / led

\section{Experience}
    \resumeSubHeadingListStart

        \resumeSubheading
            {Bioinformatics Analyst II}
            {Fred Hutchinson Cancer Center -- Seattle, WA}
            {Public Health Sciences Division - Matsen Lab}
            {Jan 2021 - Current}
            \resumeItemListStart

                \resumeItem{Algorithmic implimentation on mega-genomic datasets}
                    {
                        Implemented complex algorithm for 
                        estimating mutational fitness trajectories through time for 
                        every possible codon-mutation that may have occurred during 
                        the evolution of SARS-CoV-2.
                        this was achieved by applying novel methodologies with
                        the filtering and wrangling of a MAT-formatted
                        phylogeny representing the history of $>13$ million empirical viral genomes.
                        Using the Snakemake workflow manager, I integrated this
                        with various other upstream quality control filtering, 
                        and downstream analysis to produce the results driving an ongoing study.
                    }
                \resumeItem{Modeling Deep Mutational Scanning (DMS) datasets}
                    {
                        Presented with some equations from a colleague, 
                        I developed a novel approach to jointly modeling 
                        multiple distinct DMS experimental datasets.
                        The resulting implementation was able to accurately identify 
                        shifts in effect between respective mutations in the experiments -- in the form 
                        of parameter inference values from the proximal gradient fitting process \cite{Haddox_2023}.
                        See the \hyperref[multidms]{multidms} section for 
                        more about the respective JAX-based python package I created.
                    }

                \resumeItem{Genomic surveillance of viral escape}
                    {
                        By extending existing core Nextstrain builds (Snakemake builds), 
                        I created public Nextstrain pages for viruses such as SARS-CoV-2, Lassa, and Zika
                        which provide estimates of viral escape to monoclonal antibodies of interest.
                        This work ultimately has provided analysis tools for various studies,
                        and surveillance of viral mutation that inform the 
                        potential effectiveness of these antibodies as therapeutics.
                    }

                \resumeItem{Bioinformatics for studying affinity maturation in germinal centers}
                    {
                        Tasked with scaling up, automating, and extending
                        a rough bioinformatics pipeline for processing the raw sequence data 
                        from a novel wet-lab protocol,
                        I created an efficient Nextflow pipeline with many custom data processing steps
                        to produce feature-rich phylogeny's for more than $100$ individual sets of clonal families 
                        representing the evolution of B-cell lineages through the affinity maturation process
                        in germinal centers.
                    }

                \resumeItem{Administrative data management}
                    {
                        Contributed to the documentation and execution of best-practice
                        management and storage of sensitive lab data in various computational 
                        environments including HPC file systems and AWS S3 buckets.
                    }

                \resumeItem{Collogue consultation}
                    {
                        Provided various degrees of guidance and education to 
                        undergraduate interns, graduate students, postdocs, and various other faculty around the hutch
                        with computational tasks surrounding active research projects as well as employee onboarding.
                        This includes, but is not limited to,
                        algorithmic implementation,
                        command line interface,
                        HPC clusters,
                        batch submission (SLURM managed),
                        workflow management,
                        machine learning model exploration,
                        data formatting,
                        version control, 
                        compute environments, 
                        and build systems.
                    }

                \resumeItem{Communicating results}
                    {
                        Regularly presented results and background of my research projects
                        to a range of audiences including lab meetings, collaborative meetings, and team meetings.
                    }

            \resumeItemListEnd
        
        % Fred Hutch Internship

        \resumeSubheading
            {Research Associate}
            {Fred Hutchinson Cancer Center}
            {Computational Biology -- Matsen Lab}
            {March 2020 - Dec. 2020, Seattle, WA}
            \resumeItemListStart

                \resumeItem{PhIP-Seq software and analysis}
                    {
                        Developed a suite of software tools for
                        analysis of raw Phage ImmunoPrecipitation Sequencing
                        (PhIP-Seq) data that consisted of;
                        a python package for modeling and computing enrichment statistics,
                        an interactive application for visualizing results of the processed data,
                        and an end-to-end (raw data $\rightarrow$ results) 
                        Nextflow pipeline for encapsulating the 
                        common analysis workflows in a portable and reproducible manner.
                        In addition to informing best-practices and future experiments carried out in wet lab,
                        This software was used to process data and produce results reported in 
                    }

                \resumeItem{Neural Networks for global epistasis modeling}
                    {
                        Worked with a graduate student to implement a global epistasis
                        modeling software package using the PyTorch library.
                    }

                \resumeItem{Team Organization}
                	{
                        Set up and let team meetings to brainstorm, 
                        and set short term goals for all team members 
                        involved with more then 4 distict  projects.
                    }

            \resumeItemListEnd

        % GE Position

        \resumeSubheading
            {Graduate Educator \& Researcher}
            {Eugene, OR}
            {Biology Department -- University of Oregon Graduate School}
            {Aug 2019 - March 2020}
            \resumeItemListStart

                \resumeItem{Teaching}
                    {
                        Lectured and proctored a graduate level class focused around Python/Bash scripting,
                        Unix file formatting/parsing, and other algorithmic thinking activities. Additionally, I
                        constructed Docker containers and a Jupyter notebook autograde pipeline for 
                        students to have a unified environment for Python programming. Hosted office hours, and graded 
                        $\approx 100$ assignments/exams per week with feedback.
                    }

                \resumeItem{Deep Learning \& Simulation}
                    {
                        For my graduate research project, I wrote a simulator to model florescent immuno-labeled 
                        images. These simulations provided the dataset for training a convolutional neural network 
                        which can be used for prediction of synapse location in empirical data.
                    }

            \resumeItemListEnd

        % KERN / RALPH LAB

        \resumeSubheading
            {Scientific Programmer}
            {Eugene, OR}
            {Institute of Ecology and Evolution -- Ralph \& Kern Co-Lab}
            {June 2018 - June 2019}
            \resumeItemListStart

                \resumeItem{Software Engineering}
                    {
                        Developed for tskit$^{S5}$, stdpopsim$^{S3,P4}$. tsencode$^{S1}$, ReLERNN$^{S4,P3}$. 
                        Leveraged continual unit, coverage, and format testing 
                        to collaborate with multiple collogues for integration of
                        novel methods.
                    }

                \resumeItem{Experimentation}                
                    {
                        Quantified simulated data for training deep 
                        learning models with direct applicability to empirical data.
                        Assisted Ph.D. students, post-docs, and faculty in the lab setting with 
                        Python scripting, software installation, version control, cluster interface,
                        and statical analysis.
                    }

                \resumeItem{Scalable Data Analysis}
                    {
                        Constructed Snakemake pipeline for analysis of PopSim$^{P4}$ framework for manuscript results section.
                        This workflow produced results outlining a comparison of population demography inference tools while
                        controlling for testing data.
                    }

            \resumeItemListEnd

        % RALPH / UNDERGRADUATE WORK

        \resumeSubheading
            {Undergraduate Research Assistant}
            {Eugene, OR}
            {Mathematics Department -- University of Oregon}
            {June 2017 - June 2018}
            \resumeItemListStart

                \resumeItem{Algorithmic implementation of tree sequence recording}
                    {
                        Implemented tree sequence recording algorithm$^{P2}$ (in C++) 
                        into widely used population genetic simulation software, SLiM$^{S6}$
                    }

                \resumeItem{Benchmarking Algorithmic Implementation}
                    {
                        Benchmarked the implementation of a novel algorithm$^{P2,P5}$.              
                        Utilized cluster resourced to compare large scale simulations and measure performance across
                        a range of metrics including memory allocation and user runtime.
                    }

                \resumeItem{Simulation studies of rapid evolution in stickleback populations}
                    {
                        Developed and analyzed simulation models to study migration patterns involved in rapid evolution 
                        of stickleback fish on the Alaskan coast$^{P1}$. This project involved the construction of
                        a complex simulation model as well as Python scripts to analyze resulting data.
                    }

            \resumeItemListEnd
 
       % GRADING

        \resumeSubheading
            {Discrete Math Course Grader}
            {Eugene, OR}
            {Mathematics Department -- University of Oregon}
            {June 2016 - June 2017}
            \resumeItemListStart

                \resumeItem{Grading}
                    {
                        Graded and constructed feedback for $\approx 100$ 
                        assignments per week focused on proofs, set theory, and combinatorics.
                    }
                
            \resumeItemListEnd

    \resumeSubHeadingListEnd

%-------- Software -----------------

\section{Highlighted Software Projects}
\resumeSubHeadingListStart

    \resumeSoftwareSubheading{multidms}
        {
            \href{https://matsengrp.github.io/multidms/}{\faGlobe}
            \href{https://www.biorxiv.org/content/10.1101/2023.07.31.551037v1}{\faNewspaper}
            \href{https://github.com/matsengrp/multidms}{\faCode}
        }
        {
            Using google's JAX library, this open-source python package 
            provides the utilities for creating and jointly fitting 
            a model to multiple distinct Deep Mutational Scanning datasets.
            These models leverage Lasso regularization techniques to
            identify differences between those datasets -- in the form 
            of the resulting parameter inference values after proper fitting.
            The interface provides a number of tools which help to easily prepare data
            initialize models, fit, out-of-sample prediction, 
            as well as interactively visualizing 
            the resulting parameter values of interest.
        }            

\resumeSubHeadingListEnd
%\item [S0.] \href{https://github.com/popgensims/analysis/tree/master/n_t}{Analysis pipeline for population demography inference $^\dagger$} \space \space
%        A pipeline to analyze the use of a python package popsim$^{S3}$. 
%        This pipeline is the primary infrastructure to exemplify the importance of standardized simulation 
%        data for benchmarking of novel methods.
%        
%        \space \url{https://github.com/popgensims/analysis}
%
%\item [S1.] \href{https://github.com/jgallowa07/tsencode}{TS Encode $^\dagger$} \space \space 
%        This interface takes tskit$^{S5}$ \textit{tree sequence} objects 
%        and prepares them into 3D tensors to be used for machine learning or
%        visualization of population genetic processes. 
%        
%        \space \url{https://github.com/jgallowa07/tsencode}
%
%\item [S3.] \href{https://github.com/popgensims/stdpopsim}{stdpopsim $^\dagger$} \space \space
%        A population genetic modeling framework which provides
%        users access to a range of standardized simulation models. 
%        
%        \space \url{https://github.com/popgensims/stdpopsim}
%
%\item [S4.] \href{https://github.com/kern-lab/ReLERNN}{ReLERNN $^\dagger$} \space \space
%        Utilizing deep learning, this python interface allows users to input a genotype matrix, 
%        to infer the landscape of recombination with as little as four haploid samples. 
%
%        \space \url{https://github.com/kern-lab/ReLERNN}
%
%\item [S5.] \href{https://github.com/tskit-dev}{tskit - msprime} \space \space
%        tskit is a plethora of tools which rely on a cutting edge, tabular data structure, 
%        storing a forest of genealogical trees, dubbed \textit{TreeSequences}, relating all individuals at every genomic interval. 
%        msprime utilizes tskit for coalescent simulations.
%        
%        \space \url{https://github.com/tskit-dev}
%
%\item [S6.] \href{https://messerlab.org/slim/}{SLiM} \space \space
%        SLiM is a unique and extensive wright-fisher simulation software written with a C++ back end and 
%        interfaced with a specialized language, Eidos.
%        
%        \space \url{https://messerlab.org/slim/}
%
%\item [S7.] \href{https://play.google.com/store/apps/details?id=com.Nighthawks.SquirrelSuiter&hl=en_US}{Squirrel Suiter $^\dagger$} 
%        \space \space A fun 3D ``infinite flyer" I made with some friends in undergrad! 
%        
%        \space \url{https://play.google.com/store/apps/details?id=com.Nighthawks.SquirrelSuiter&hl=en_US}
%
%\item [S7.] \href{https://github.com/matsengrp/phippery}{Phippery $^\dagger$} 
%        \space \space A set of tools to organize and query data resulting from PhIP-Seq experiments, most commonly 
%        referred to as the enrichment matrix.
%        This package heavily relies on the Xarray data structure and provides convenient tools to compute
%        various forms of normalization, modeling, and other common analysis methods.
%        
%        \space \url{https://github.com/matsengrp/phippery}
%
%\item [S8.] \href{https://github.com/matsengrp/phip-flow}{PhIP-Flow $^\dagger$}
%        \space \space A generalized Nextflow pipeline for aligning and organizing PhIP-Seq NGS data for downstream analysis.
%        The pipeline provides an intuitive set of input including sample and peptide metadata, and the relevant NGS files,
%        and produces a Xarray dataset containing the enrichment matrix tied to the respective metadata for each peptide and sample using unique
%        identifiers as the coordinate dimensions. 
%
%        \space \url{https://github.com/matsengrp/phip-flow}
%
%\item [S9.] \href{https://github.com/fhcrc/seqmagick}{SeqMagik}
%        \space \space We often have to convert sequence files between formats and do little manipulations on them, 
%        and it's not worth writing scripts for that. 
%        Seqmagick is a kickass little utility to expose the file format conversion in BioPython in a convenient way
%
%        \space \url{https://github.com/fhcrc/seqmagick}
%
%
%\end{enumerate}

%-------- Primary Author Publications -----------------
\bibliographystyleprim{unsrt_abbrv_custom}
\nociteprim{*}
\small\bibliographyprim{first}

%-------- Supporting Author Publications -----------------
\bibliographystylesupp{unsrt_abbrv_custom}
\nocitesupp{*}
\bibliographysupp{supporting}

%-------- Professional References -----------------
\section{Professional References}

\resumeSubHeadingListStart
    
    \resumeSubItem {Dr. Erick Matsen}
        {Principal Investigator -
        %\href{https://www.fredhutch.org/en/faculty-lab-directory/matsen-frederick.html}
        %{Fred Hutchinson Cancer Research Center} -
        matsen@fredhutch.org}
    \resumeSubItem {Dr. Leslie Coonrod}
        {Associate Director of Master's Program -
        %\href{https://accelerate.uoregon.edu/leslie-coonrod}
        %{Bioinformatics and Genomics Master's Program} -
        coonrod@uoregon.edu}
    \resumeSubItem {Dr. Stacey Wagner}
        {Director of Master's Program -
        %\href{https://accelerate.uoregon.edu/stacey-wagner}
        %{Director of Life Sciences Program} - 
        sdwagner@uoregon.edu}
    \resumeSubItem {Dr. Bill Cresko}
        {Research Advisor -
        %\href{https://creskolab.uoregon.edu/}
        %{Cresko Lab} - 
        wcresko@uoregon.edu}
    \resumeSubItem {Dr. Peter Ralph}
        {Principal Investigator -
        %\href{https://pages.uoregon.edu/plr/}
        %{Ralph Lab} - 
        plr@uoregon.edu}
    \resumeSubItem {Dr. Andrew Kern}
        {Principal Investigator -
        %\href{http://www.kernlab.org/} 
        %{Kern Lab} -
        adkern@uoregon.edu}
    \resumeSubItem {Dr. Ben Haller}
        {Research Advisor - 
        %\href{http://benhaller.com/}
        %{Website} - 
        haller@mac.com}
\resumeSubHeadingListEnd

%-------- Supporting Author Publications -----------------
\section{Achievments}

\resumeSubHeadingListStart
    \resumeSubItem{Graduate Thesis Project}
    {
        Machine learning for identifying synapses.
    } % TODO \href{}{poster}:  
    \resumeSubItem{Undergraduate Thesis Project}
    {
        Speeding up the Tortoise: a case study in the optimization of forward-moving evolutionary simulations
    }\href{https://www.cs.uoregon.edu/Reports/UG-201806-Galloway.pdf}{University Arxiv}: https://www.cs.uoregon.edu/Reports/UG-201806-Galloway.pdf
    \resumeSubItem{Community Service}
    {
        Built trail for Montana Conservation Corps and acquired 300+
        hours community service
    }
\resumeSubHeadingListEnd

\end{document}
