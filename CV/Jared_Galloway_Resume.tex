%-------------------------
% Resume in Latex
% Author : Jared Galloway
% License : MIT
%------------------------

\documentclass[letterpaper,11pt]{article}

\usepackage{bibentry}

\usepackage{latexsym}
\usepackage[empty]{fullpage}
\usepackage{titlesec}
\usepackage{marvosym}
\usepackage[usenames,dvipsnames]{color}
\usepackage{verbatim}
\usepackage{enumitem}
\usepackage[hidelinks]{hyperref}
\usepackage{fancyhdr}
\usepackage[english]{babel}

\pagestyle{fancy}
\fancyhf{} % clear all header and footer fields
\fancyfoot{}
\renewcommand{\headrulewidth}{0pt}
\renewcommand{\footrulewidth}{0pt}

% Adjust margins
\addtolength{\oddsidemargin}{-0.5in}
\addtolength{\evensidemargin}{-0.5in}
\addtolength{\textwidth}{1in}
\addtolength{\topmargin}{-.5in}
\addtolength{\textheight}{1.0in}

\usepackage{hyperref}
\hypersetup{
    %colorlinks=true,
    %linkcolor=blue,
    %filecolor=magenta,      
    %urlcolor=blue,
}
\urlstyle{same}
\let\oldhref\href
\renewcommand{\href}[2]{\oldhref{#1}{\bfseries#2}}

\raggedbottom
\raggedright
\setlength{\tabcolsep}{0in}

% Sections formatting
\titleformat{\section}{
  \vspace{-4pt}\scshape\raggedright\large
}{}{0em}{}[\color{black}\titlerule \vspace{-5pt}]

%-------------------------
% Custom commands
\newcommand{\resumeItem}[2]{
  \item\small{
    \textbf{#1}{: #2 \vspace{-2pt}}
  }
}

\newcommand{\resumeSubheading}[4]{
  \vspace{-1pt}\item
    \begin{tabular*}{0.97\textwidth}[t]{l@{\extracolsep{\fill}}r}
      \textbf{#1} & #2 \\
      \textit{\small#3} & \textit{\small #4} \\
    \end{tabular*}\vspace{-5pt}
}


\newcommand{\resumeSubItem}[2]{\resumeItem{#1}{#2}\vspace{-4pt}}

\renewcommand{\labelitemii}{$\circ$}

\newcommand{\resumeSubHeadingListStart}{\begin{itemize}[leftmargin=*]}
\newcommand{\resumeSubHeadingListEnd}{\end{itemize}}
\newcommand{\resumeItemListStart}{\begin{itemize}}
\newcommand{\resumeItemListEnd}{\end{itemize}\vspace{-5pt}}


%-------------------------------------------
%%%%%%  CV STARTS HERE  %%%%%%%%%%%%%%%%%%%%%%%%%%%%


\begin{document}

%----------HEADING-----------------
\begin{tabular*}{\textwidth}{l@{\extracolsep{\fill}}r}
  \textbf{\Large Jared G. Galloway, M.Sc.} \\
  &  \href{jaredgalloway07@gmail.com}{jaredgalloway07@gmail.com} \\
  &  \href{https://github.com/jgallowa07}{https://github.com/jgallowa07} \\
  &  \href{http://jaredgalloway.org/}{http://jaredgalloway.org/} \\
  &  +1-406-579-6768
\end{tabular*}

%-----------PERSONAL STATEMENT--------
%\section{Personal Statement}
%Utilizing a backgound in computer science and an interest in biological mechanisms, i
%I have focused on using machine learning and simulation techniques to 
% understand how evolutionary forces influence the variation we see in species’ genomes. 
% Moving forward I hope to leverage new computational techniques 

%-----------EDUCATION-----------------
\section{Education}
    \resumeSubHeadingListStart
        \resumeSubheading
            {Master of Science in Biology}{University of Oregon}
            {Bioinformatics and Genomics Program}{June 2019 -- Exp. Dec. 2020}
        \resumeSubheading
            {Bachelor of Science in Computer and Information Science}{University of Oregon}
            {Departmental Honors - Minor in Product Design}{Aug 2013 -- July 2018}
    \resumeSubHeadingListEnd

%--------PROGRAMMING SKILLS------------
\section{Technical Skills}

\resumeSubHeadingListStart
    \resumeSubItem{Programming Languages}{Python, C/C++, Bash, R, Java, Javascript}
    \resumeSubItem{Technologies}
        {Unix, Git, Docker, Conda, 
        Cmake, Snakemake, Numpy, 
        Keras/Tensorflow, Vim, \LaTeX, HPCs}
    \resumeSubItem{Mathematics}{Advanced and Bayesian Statistics, Linear Algebra, Discrete Math, Calculus}
\resumeSubHeadingListEnd

%-----------EXPERIENCE-----------------
\section{Experience}
    \resumeSubHeadingListStart
        
        % Fred Hutch

        \resumeSubheading
            {Research Associate}
            {Seattle, WA}
            {Computational Biology -- Fred Hutchinson Cancer Research Center}
            {March 2020 - Current}
            \resumeItemListStart

                \resumeItem{Teaching}
                    {Lectured and proctored a graduate level class focused around Python/Bash scripting,
                    Unix file formatting/parsing, and other algorithmic thinking activities. Additionally, I
                    constructed Docker containers and a Jupyter notebook autograde pipeline for 
                    students to have a unified environment for Python programming. Hosted office hours, and graded 
                    $\approx 100$ assignments/exams per week with feedback.}

                \resumeItem{Deep Learning \& Simulation}
                    {For my graduate research project, I wrote a simulator to model florescent immuno-labeled 
                    images. These simulations provided the dataset for training a convolutional neural network 
                    which can be used for prediction of synapse location in empirical data.}

            \resumeItemListEnd

        % GE Position

        \resumeSubheading
            {Graduate Educator \& Researcher}
            {Eugene, OR}
            {Biology Department -- University of Oregon Graduate School}
            {Aug 2019 - March 2020}
            \resumeItemListStart

                \resumeItem{Teaching}
                    {Lectured and proctored a graduate level class focused around Python/Bash scripting,
                    Unix file formatting/parsing, and other algorithmic thinking activities. Additionally, I
                    constructed Docker containers and a Jupyter notebook autograde pipeline for 
                    students to have a unified environment for Python programming. Hosted office hours, and graded 
                    $\approx 100$ assignments/exams per week with feedback.}

                \resumeItem{Deep Learning \& Simulation}
                    {For my graduate research project, I wrote a simulator to model florescent immuno-labeled 
                    images. These simulations provided the dataset for training a convolutional neural network 
                    which can be used for prediction of synapse location in empirical data.}

            \resumeItemListEnd

        % KERN / RALPH LAB

        \resumeSubheading
            {Scientific Programmer}
            {Eugene, OR}
            {Institute of Ecology and Evolution -- Ralph \& Kern Labs}
            {June 2018 - June 2019}
            \resumeItemListStart

                \resumeItem{Software Engineering}
                    {Devolped for tskit$^{S5}$, stdpopsim$^{S3,P4}$. tsencode$^{S1}$, ReLERNN$^{S4,P3}$. 
                    Leveraged continual unit, coverage, and format testing 
                    to collaborate with multiple collogues for integration of
                    novel methods.}

                \resumeItem{Experimentation}                
                    {Quantified simulated data for training deep 
                    learning models with direct applicability to empirical data.
                    Assisted Ph.D. students, post-docs, and faculty in the lab setting with 
                    Python scripting, software installation, version control, cluster interface,
                    and statical analysis.}

                \resumeItem{Scalable Data Analysis}
                    {Constructed Snakemake pipeline for analysis of PopSim$^{P4}$ framework for manuscript results section.
                    This workflow produced results outlining a comparison of population demography inference tools while
                    controlling for testing data.}

            \resumeItemListEnd

        % RALPH / UNDERGRADUATE WORK

        \resumeSubheading
            {Undergraduate Research Assistant}
            {Eugene, OR}
            {Mathematics Department -- University of Oregon}
            {June 2017 - June 2018}
            \resumeItemListStart

                \resumeItem{Algorithmic Implementation}
                    {Implemented tree sequence recording algorithm$^{P2}$ (in C++) 
                    into widely used population genetic simulation software, SLiM$^{S6}$}

                \resumeItem{Benchmarking Algorithmic Implementation}
                    {Benchmarked the implementation of a novel algorithm$^{P2,P5}$.              
                    Utilized cluster resourced to compare large scale simulations and measure performance across
                    a range of metrics including memory allocation and user runtime.}

                \resumeItem{Modeling for Population Genetics}
                    {Developed and analyzed simulation models to study migration patterns involved in rapid evolution 
                    of stickleback fish on the Alaskan coast$^{P1}$. This project involved the construction of
                    a complex simulation model as well as Python scripts to analyze resulting data.}

            \resumeItemListEnd
 
       % GRADING

        \resumeSubheading
            {Discrete Math Course Grader}
            {Eugene, OR}
            {Mathematics Department -- University of Oregon}
            {June 2016 - June 2017}
            \resumeItemListStart

                \resumeItem{Grading}
                    {Graded and constructed feedback for $\approx 100$ 
                    assignments per week focused on proofs, set theory, and combinatorics.}
                
            \resumeItemListEnd

    \resumeSubHeadingListEnd

%-------- Publications -----------------
\section{Software}

\begin{enumerate}

\item [S2.] \href{https://github.com/popgensims/analysis/tree/master/n_t}{Analysis Pipeline $^{\dagger}$} \space \space
        A pipeline to analyze the use of a python package popsim$^{S3}$. 
        This pipeline is the primary infrastructure to exemplify the importance of standardized simulation 
        data for benchmarking of novel methods.
        
        \space \url{https://github.com/popgensims/analysis}

\item [S1.] \href{https://github.com/jgallowa07/tsencode}{TS Encode $^{\dagger}$} \space \space
        This interface takes tskit$^{S5}$ \textit{tree sequence} objects 
        and prepares them into 3D tensors to be used for machine learning or
        visualization of population genetic processes. 
        
        \space \url{https://github.com/jgallowa07/tsencode}

\item [S3.] \href{https://github.com/popgensims/stdpopsim}{stdpopsim $^{\dagger}$} \space \space
        A population genetic modeling framework which provides
        users access to a range of standardized simulation models. 
        
        \space \url{https://github.com/popgensims/stdpopsim}

\item [S4.] \href{https://github.com/kern-lab/ReLERNN}{ReLERNN $^{\dagger}$} \space \space
        Utilizing deep learning, this python interface allows users to input a genotype matrix, 
        to infer the landscape of recombination with as little as four haploid samples. 

        \space \url{https://github.com/kern-lab/ReLERNN}

\item [S5.] \href{https://github.com/tskit-dev}{tskit - msprime} \space \space
        \testbf{tskit} is a plethora of tools which rely on a cutting edge, tabular data structure, 
        storing a forest of genealogical trees, dubbed \textit{TreeSequences}, relating all individuals at every genomic interval. 
        msprime utilizes tskit for coalescent simulations.
        
        \space \url{https://github.com/tskit-dev}

\item [S6.] \href{https://messerlab.org/slim/}{SLiM} \space \space
        SLiM is a unique and extensive wright-fisher simulation software written with a C++ back end and 
        interfaced with a specialized language, Eidos.
        
        \space \url{https://messerlab.org/slim/}

\item [S7.] \href{https://play.google.com/store/apps/details?id=com.Nighthawks.SquirrelSuiter&hl=en_US}{Squirrel Suiter $^{\dagger}$} 
        \space \space A fun 3D ``infinite flyer" I made with some friends in undergrad! 
        
        \space \url{https://play.google.com/store/apps/details?id=com.Nighthawks.SquirrelSuiter&hl=en_US}


\end{enumerate}

%-------- Publications -----------------
\section{Publications \& Highlighted Achievements}
%\nobibliography{ref}
%\bibliographystyle{unsrt}
%\nobibliography{ref}

\begin{enumerate}
\item [P1.] \textbf{Jared G. Galloway} $^{\dagger}$, William A. Cresko, Peter L. Ralph. A few stickleback suffice for the transport
of alleles to new lakes. \textit{Published in} 
\href{https://www.g3journal.org/content/early/2019/12/04/g3.119.400564}{G3: Genes, Genomes, and Genetics}, Oct 2019.
\href{https://www.biorxiv.org/content/10.1101/713040v1.abstract}{BioArxiv}: 10.1101/713040v1

\item [P2.] Benjamin C. Haller, \textbf{Jared G. Galloway}, Jerome Kelleher, Philipp W. Messer, and Peter L. Ralph.
Tree-sequence recording in SLiM opens new horizons for forward-time simulation of whole genomes. 
\textit{Published in} 
\href{https://onlinelibrary.wiley.com/doi/abs/10.1111/1755-0998.12968}{MER: Molecular Ecology Resources}, 19(2):552-566. Jan 2019. 
\href{https://www.biorxiv.org/content/10.1101/407783v1}{BioArxiv}: 10.1101/407783v1 

\item [P3.] Jeffrey R. Adrion, \textbf{Jared G. Galloway} $^{\dagger}$, Andrew D. Kern. Inferring the landscape of recombination 
using recurrent neural networks. \textit{Accepted in} 
\href{https://academic.oup.com/mbe}{Molecular Biology and Evolution}, Jan 2020.
\href{https://www.biorxiv.org/content/10.1101/662247v1.abstract}{BioArxiv}: 10.1101/662247v1

\item [P4.] \textbf{Jared G. Galloway} $^{\dagger}$, PopSim Consortium:
A community-maintained standard library of population genetic simulations.
\textit{Under Review at}
\href{https://elifesciences.org/?gclid=CjwKCAiA6vXwBRBKEiwAYE7iS0LA_KboY5NjoOVJAMq06BEUSsqPFV9R1GA1NUUIgYw2XgTiv1fUxhoC3xYQAvD_BwE}{eLife}, Jan 2020.
\href{https://www.biorxiv.org/content/10.1101/2019.12.20.885129v1}{BioArxiv}: 10.1101/2019.12.20.885129 

\item [P5.] \textbf{Jared G. Galloway} $^{\dagger}$. Speeding up the Tortoise: a case study in the optimization of forward-moving
evolutionary simulations. \textit{Undergraduate Thesis - CIS dept. University of Oregon.} 
\href{https://www.cs.uoregon.edu/Reports/UG-201806-Galloway.pdf}{University Arxiv}: https://www.cs.uoregon.edu/Reports/UG-201806-Galloway.pdf

\item [P6.] $300+$ Hours community service building trail for Montana Conservation Corps. 
\href{https://mtcorps.org/}{website}: https://mtcorps.org/

\begin{center}
$^{\dagger}$ primary/significant author
\end{center}

\end{enumerate}

%\section{Achievments}
%
%\resumeSubHeadingListStart
%    \resumeSubItem{Graduate Thesis project}{Machine learning for identifying synapses}{blah}
%    \resumeSubItem{}{}
%    \resumeSubItem{Community Service}{Built trail for Montana Conservation Corps and acquired 300+
%        hours community service}
%\resumeSubHeadingListEnd


%-------- REFERENCES -----------------
%\section{References}
%
%\resumeSubHeadingListStart
%    \resumeSubItem {Dr. Bill Cresko}
%        {Research Advisor -
%        \href{https://creskolab.uoregon.edu/}
%        {Cresko Lab} - 
%        wcresko@uoregon.edu}
%    \resumeSubItem {Dr. Peter Ralph}
%        {Research Advisor -
%        \href{https://pages.uoregon.edu/plr/}
%        {Ralph Lab} - 
%        plr@uoregon.edu}
%    \resumeSubItem {Dr. Andrew Kern}
%        {Research Advisor -
%        \href{http://www.kernlab.org/} 
%        {Kern Lab} -
%        adkern@uoregon.edu}
%    \resumeSubItem {Dr. Ben Haller}
%        {Research Advisor - 
%        \href{http://benhaller.com/}
%        {Website} - 
%        haller@mac.com}
%\resumeSubHeadingListEnd
%
%-------------------------------------------
\end{document}
















